\documentclass{article}
%\usepackage[superscript]{cite}
\usepackage[pdftex]{graphicx}
\usepackage{sidecap}
\usepackage{caption}
\usepackage{subcaption}
\usepackage{amsmath}
\usepackage{amsfonts}
\usepackage{tikz}
\usepackage{pgfplots}
\usepackage[colorlinks=true,linkcolor=blue,citecolor=blue]{hyperref}
\usepackage{fullpage}
%\usepackage{natbib}
\usepackage[utf8]{inputenc}
\usepackage[english]{babel}
\newcommand{\angstrom}{\text{\normalfont\AA}}


\begin{document}
%\section{Phase I}
%We want to define the scope of the current project such that it is something we feel we can accomplish, and be relatively complete in our data collection.
%


\section{Procedure}
%Three primary tasks 
%\begin{itemize}
%\item \textbf{Data Research} The person assigned to data research will search the literature to find out \underline{what data has been taken for a given nova.} 
%\item \textbf{Data Acquisition} The person assigned to retrieving the data: this can mean either downloading the data or contacting the author of the original paper.
%\item \textbf{Data Importation} The person assigned to incorporating the data into the ONC software.
%\end{itemize}
Three step procedure:
\begin{itemize}
\item \textbf{Data Research} Data research involves searching the literature to find out what data has been taken for a given nova.
\item \textbf{Data Acquisition} Retrieving the data. This can mean either downloading the data or contacting the author of the original paper.
\item \textbf{Data Importation} Incorporating the data into the ONC software.
\end{itemize}


\section{Ticket System}
We will use the ticket system: each new piece of data will be assigned a ticket. This serves a two-fold purpose: 
\begin{enumerate}
\item  It ensures that we are, at the very least, aware of all of the data that has been used in published articles for a given nova.
\item It ensures that, from the start, we maintain an accurate provenance for each piece of data, as well as complete meta-data.
\end{enumerate}

Tickets will be initially created in the data research step, and fields will be filled in during both the data research and data acquisition steps. A complete ticket means that the data is ready to be incorporated into the ONC.

\subsection{When to Make a New Ticket}
There are several factors that determine when a new ticket should be generated
\begin{enumerate}
\item \textbf{Reference} A single ticket should never have multiple references. If the same group took data on the same nova, using the same telescope/instrument, but published part of it in a different article, it will still get its own ticket.
\item \textbf{Nova} If a single reference has data for multiple novae, every novae from that reference will get its own ticket. 
\item \textbf{Data Types} If a single paper publishes both photometry and spectra, each of these should get their own ticket. This is because the meta-data associated with photometry and spectra are different. 
\item \textbf{Wavelength Regime} If a single paper publishes both radio and optical photometry, each of these should get their own ticket. This is done mostly for the ease of bookkeeping.
\end{enumerate}


\subsection{When NOT to Make a New Ticket}
There are also several cases that may seem like they would necessitate different tickets but do not.
\begin{enumerate}
\item \textbf{Telescopes/Observers/Filter Systems} Most collection of spectra and photometry are a compilation from many different telescopes, observers, and filter systems (e.g. AAVSO). Giving each one their own ticket would end up generating a flood of tickets for each data collection. So, instead, we have included the ability to specify different Telescopes/Observers/Filter Systems in the data file, by allowing the user to specify columns within the file that provide information on these differences.
\end{enumerate}

%There are two factors that determine 

\subsection{Naming Convention}
To make parsing/understanding individual tickets easier, we will use a specific naming convention for them. Every ticket should use the following naming convention:\\

\noindent
\begin{Large}
$<$\texttt{Nova Name}$>$\_$<$\texttt{First Author}$>$\_$<$\texttt{Wavelength Regime}$>$\_$<$\texttt{Data Type}$>$.txt
\end{Large}
\\
Where data type can be photometry, spectra, or image.

So, for instance, if the ticket was for the radio photometry for FH Ser taken from Hjellming et al. (1979) the ticket name would be:
\\

\noindent
\begin{Large}
FHSer\_Hjellming\_Radio\_Photometry.txt
\end{Large}


%\subsection{Multiple Datasets from a Single Source}




\newpage
\appendix
\section{Example: Ticket for FH Ser Radio (Photometry) Data}

\begin{itemize}
\item \textbf{OBJECT NAME:} FHSer
\item \textbf{TIME UNITS:} Days
\item \textbf{FLUX UNITS:} mJy
\item \textbf{FLUX ERROR UNITS:} mJy
\item \textbf{FILTER SYSTEM:} NA
\item \textbf{MAGNITUDE SYSTEM:} NA
\item \textbf{WAVELENGTH REGIME:} Radio
\item \textbf{TIME SYSTEM:} MJD
\item \textbf{ASSUMED DATE OF OUTBURST:}
%\item \textbf{UPPER LIMIT:} 
\item \textbf{TELESCOPE:} NA
\item \textbf{INSTRUMENT:} NA
%\item \textbf{OBSERVATORY:} VLA
\item \textbf{OBSERVER:} Hjellming, R.
\item \textbf{REFERENCE:} 1979AJ.....84.1619H
\item \textbf{DATA FILENAME:}
\item \textbf{TIME COLUMN NUMBER:}
\item \textbf{FLUX COLUMN NUMBER:} 
\item \textbf{FLUX ERROR COLUMN NUMBER:} 
%\item \textbf{FILTER \emph{or} FREQUENCY \emph{or} ENERGY RANGE COLUMN NUMBER:} 
\item \textbf{FILTER/FREQUENCY/ENERGY RANGE COLUMN NUMBER:} 
\item \textbf{UPPER LIMIT FLAG COLUMN NUMBER:} 
\item \textbf{TELESCOPE COLUMN:} NA
\item \textbf{INSTRUMENT COLUMN:} NA
\item \textbf{OBSERVER COLUMN:} NA
\item \textbf{TICKET STATUS:} Waiting for acquisition.
\end{itemize}

%Hopefully we will be able to implement multiple phases of 


%
%\section*{Guiding Principles}
%These are taken from James' paper. Since we are already using James' format, some of the principles are (essentially) built in for us, so I've reordered them based on what I think will be most important for us to keep in mind as we move along.
%
%\begin{itemize}
%\item \textbf{Accuracy and Accountability}: We need to make sure that we include \emph{all} relevant information for any given dataset
%\item \textbf{Completeness and Persistence}: We need to hunt down all possible data sources with a fervent determination. Our goal should always be to have \emph{everything that is publicly available}. There is no such thing as a dataset too small.
%\item \textbf{Community Driven}: The more we can get people involved and invested, the easier our job will be.
%\item \textbf{Accessibility}
%\item \textbf{Integration}
%
%\end{itemize}
%
%
%
%\section*{Tasks}
%\begin{itemize}
%\item \textbf{Data Collection}: This will be an ongoing process, and falls under the purview of the data acquisition leader. Here is a list of tasks that we can do to get started.
%\begin{enumerate}
%\item Contact Fred Walter and Ulisse Munari, who already have extensive datasets for southern and northern novae, respectively.  
%\item Pull from AAVSO. I believe we can do this in an automated way.
%\item Get radio data, both from e-NOVA and from literature.
%\end{enumerate}
%\item \textbf{Data Formatting}: This falls under the purview of the software leader
%\begin{enumerate}
%\item Determine what format James' software is going to be expecting 
%\item Write software that will reformat the data so that it is compatible with James' software. 
%\end{enumerate}
%\item \textbf{Quality Control}: This falls under the purview of the quality control leader
%\begin{enumerate}
%\item Make sure that all data can be traced back to its origin
%\item Make sure that all measured values have associated uncertainties. I'm not sure what we should do with data that doesn't have uncertainties...
%\end{enumerate}
%\end{itemize}
%
%
%
%%\section{Specific Leader Roles}
%%\begin{itemize}
%%\textbf{Data Collection}
%%\textbf{Software Design}
%%\textbf{Quality Control}
%%\end{itemize}
%
%%This isn't to say that the entire burden will be on
%
%
%
%
%
%





\end{document}